\section{Definition}

\subsection{Traffic Model}
The traffic is flowing in two ways on two intersecting roads. Based on the
direction of car flows, the traffic model is comprised of four independent
one-way flows, denoted as West-East, East-West, North-South and South-North. For
example, in the North-South flow, all the cars travel in one direction, that is,
from north to south.

The competing flows of the traffic are controlled by a set of traffic lights. As
there are only two intersecting main roads (regardless the directions), two
mutually exclusive traffic lights are sufficient. They are denoted as lightWE
(which controls West-East and East-West), and lightNS, for North-South and
South-North traffic flows. There are three signals, namely RED, AMBER and GREEN.
A car should stop if there is an AMBER or RED signal, and the cars behind
it form a stationary queue. In addition, the lights can not change consecutively in
less than three timesteps.

Each flow is modelled as a standalone road. The road is 100 units in length
(from 0 to 99), and intersects with the roads in the perpendicular directions at
Unit 49 and 50. The traffic light is positioned at road intersections. At each
timestep, one car enters the road at Position 0, with the probability that
equals to the traffic intensity (e.g. 15\%). Cars are removed from the road when
they reach the boundary (i.e., beyond unit 99).

\subsection{State Representation}
The traffic state consists of the light's state and the car's state. As the
light signals are mutually exclusive, there are only four possible states,
namely (GREEN, RED), (AMBER, RED), (RED, GREEN) and (RED, AMBER), denoted as 0,
1, 2 and 3 respectively. Moreover, the value of light delay since last change is
0-3. Hence, the state space of lights is 4 x 4.

However, defining the car's state is less straightforward. The occupancy of each
unit by a car can be marked as 0 or 1. As there are 48 units before the traffic
lights on each road, the state space of cars can be as large as $2^{192}$.
Therefore, we propose three variants of state representation to maintain the
space in a feasible size.

\subsubsection{Default State}
The Default State is described in the assignment specification. On each road,
only the position of the closest car from the intersection is counted, and only
the first 9 units from the intersection are inspected. Therefore, the position
is between 0 and 8 inclusive, and 9 denotes no cars. The space size of Default
State is $10^4$.

\subsubsection{JamNess State}
The JamNess State is a representation of how busy the traffic is on that road
(i.e. traffic jam-ness). Each car is given a weighting based on their distance
from the intersection. Cars closer to the intersection have a greater weight
than cars further away from the intersection. To calculate JamNess, we introduce
an importance value of each car, denoted as $0.5^i$, where $i$ is the position
of the car from the traffic lights (at Unit 49). For example, the importance of
the car at Unit 46 is $0.5^3$, since it is 3 positions away from the lights. The
sum of all the importance values on one road is between 0 and 0.9999999. Then,
the JamNess is calculated as $sum * 10$ and rounded down to an integer, which is
between 0 and 9 inclusive. The space size of JamNess State is $10^4$.

\subsubsection{Occupancy State}
The Occupancy State represents the presence of cars at each unit. As discussed
above, each unit is marked as 0 or 1, there are $2^{48}$ states if all the units
are counted. However, since the traffic lights can be switched every three
timesteps, we can speculate that the closest three or four positions from the
intersection are more important than the rest. Hence, we define two states,
Occ\_3 and Occ\_4, which are represented by a 3-bit and 4-bit integer
respectively, indicating the occupancy of the closest three or four positions.
For example, a binary value 100 in Occ\_3 indicates the first position is
occupied, while the second and the third are empty. A binary value 0101 in
Occ\_4 means the second and the fourth positions are occupied, whilst the rest
are not. Since there are four roads, the space size of Occ\_3 is $2^{12}$, and
that of Occ\_4 is $2^{16}$.

\subsubsection{Density State}
The Density State records the number of cars on the N closest positions from the
intersection, where N is 3 in the implementation, since the traffic lights can
only be switched every three timesteps. The number of cars in the closest three
positions is between 0 and 3, so the space size of four roads is $4^4$ or $2^8$.

There is an extension of Density State (N=3). To help the controller predict the
future traffic, we not only calculate the number of cars at the first three
positions (i.e., from Unit 48 to 46), we also calculate the car number of the
second three positions (i.e., from Unit 45 to 43). This extension is called
Density\_6, as there are six positions involved. By constrast, Density State
(N=3) is called Density\_3. The space size of this extension is $(4 * 4) ^{4}$,
or $2^{16}$.